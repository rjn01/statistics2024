\documentclass{article}
\usepackage{graphicx} % Required for inserting images


\date{November 2024}

\begin{document}

\maketitle

\section{Question 4}
\section{(a)}
To show that \(f_k\) is a probability density function, these conditions have to hold:
\[f_k(x) \geq 0, \forall x \in (0,\infty) \tag{1}\]
\[
\int_{-\infty}^{\infty} f_k(x)dx =1 \tag{2}
\]
The (1) is proved because for \(x<1\), \(f_k(x)=0\) and for \(x \geq 1\), the sign of the function only depends on \(k\), which is non-negative because it is a natural number.
The (2) is proved because: \[
\int_{-\infty}^{1} 0 dx + \int_{1}^{\infty}\frac{k}{x^{k+1}}dx=k\int_{1}^{ \infty}x^{-k-1}dx=1, \forall k>0 \tag{3}
\]
\section{(b)}
To prove the existence of the moments, we need to proof that they are finite. Therefore:
\[
\mathbb{E}[X^ℓ] = \int_{1}^{\infty}\frac{kx^ℓ}{x^{k+1}}dx=k\int_{1}^{\infty}x^{ℓ-k-1}dx \tag{4}
\]
To be finite, the integrand have to converge, which happens when \(ℓ-k-1<-1\), or \(ℓ<k\). This shows that the moments \(\mathbb{E}[X^ℓ]\) exist for \(0 \leq ℓ\leq k-1\) and not for \(ℓ\geq k\).
\section{(c)}
Each of the random variables \(X_i\) is integrable if its expected value is finite. Noting (4):\[
\mathbb{E}[X]=k\int_{1}^{\infty}x^{-k}dx \tag{5}
\]
we can say that \(\mathbb{E}[X]\) is finite when \(-k<-1\). Considering that the random variables \(X_i\) are i.i.d. by construction, the weak Law of Large Numbers suggests that the sample average \(\overline{X}_n\) converges in probability to \(\mathbb{E[X]}\) for \(k>1\). For \(k=1\), \(\mathbb{E}[X]\) does not exist.
\end{document}
